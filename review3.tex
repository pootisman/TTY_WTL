\documentclass[a4paper, titlepage, 10pt]{article}
\usepackage[T2A]{fontenc}
\usepackage[utf8]{inputenc}
\usepackage[english, russian]{babel}
\usepackage{indentfirst}
\hoffset=-2.1cm
\voffset=-1.6cm
\setlength{\parindent}{1cm}
\textwidth= 16cm
\textheight = 24cm
\begin{document}
\title{Review of the Team 4 report: \\ Properties of coplanar waveguides.}
\section{What could have been better.}
The theoretical part of the report is done pretty well. There are a few minor points to mention though. For instance, in the beggining of the part some practical information is given (what frequencies are used with this waveguide, problems arising from the use of these frequencies and etc.). It is good that people reading your report will not get blasted away by mathematical expressions right at the start, however, it is a bit strange that the title does not fully correspond to the content. Another minor note is about images in the simulation section. Maybe something is wrong with my software, but there is a gray line on the left side of the pictures. This is also a bit strange, but it does not get into my eye. But anyway, it's better to be careful while processing images.
\section{What was done flawlessly.}
The first thing that I noticed while having a short look into Your report is that the pictures of the waveguide are in grayscale. This might sound odd, but that is actually (for me at least) a good practice to use grayscale pictures where they can be used in order to support more outputs (what if the person, who is reading a report uses old laser printer?). Another nice thing that You did was changing different variables in the simulation and producing multiple dependencies of the variables. This is really shows some interesting facts about how the values of the system change depending on the parameters.
\section{Summary.}
All in all, I can't say that the report is flawless, but generally it does what it is supposed to do. It shows us how theory correlates with practice, what kind of system we are looking at and how big marginal errors are.
\end{document}
